
\indent \par \gls{dsp} is a field of applied mathematics and physics on a collection of numbers known as samples. The entire collection of these samples emulate the shape of the voltage sent to a speaker to play audio or voltage received from a microphone. Once these samples are stored digitally, they can be modified in some shape or form. Then, these samples can be sent through a piece of hardware to be sent to a speaker to play music.

For example, multiplying every sample in this collection by 0.5 before sending the signal to a speaker will cut the amplitude in half, effectively making the signal sound 6 dB quieter.

Basic \gls{dsp}, such as the above example of gain reduction, is in every device that can play audio and record it, like phones and laptops. More complex \gls{dsp} techniques are in devices like mixing consoles and software such as Digital Audio Workstations.

Mixing consoles typically have dedicated hardware known as digital signal processors. These chips are especially built for \gls{dsp}, and they're designed to do arithmetic on samples extremely quickly. Audio sample rates are typically 44.1kHz, 48kHz, 96kHz, and 192kHz, which are the number of samples per second. The more samples there are per second, the higher the frequency resolution of the audio is. At the same time, that's more computations that processors have to compute per second. At the time of writing, the industry standard sample rate is in the process of moving from 48kHz to 96kHz. The drawback of moving to 96kHz is the doubling in computation time, which this paper will offer suggestions to enhance. Digital signal processors, because they are specialized for \gls{dsp}, are incredibly fast and can handle 96kHz or 192kHz with no problem, but they are only found in specific hardware.

On a computer, the \gls{cpu} typically does all of the processing. While this chip is powerful enough to perform \gls{dsp} on large amounts of audio, the problem is that the \gls{cpu} also has to multitask and run other programs at the same time. The operating system itself of a computer is a massive, resource-intensive piece of software. Any interactive, click-and-point interface also requires resources. In addition, the way that the \gls{cpu} is designed, it is supposed to have average efficiency for all possible tasks thrown at it, which doesn't make it the most effective hardware for \gls{dsp}.

This research project aims to offload the work of \gls{dsp} onto a different chip, the \gls{gpu}, and to compare the performance results to that of a \gls{cpu}. The two chips are designed to do arithmetic and computations, but the primary difference is that a \gls{gpu} is designed to do computations in parallel where a \gls{cpu} is designed to do computations sequentially. Meaning, a \gls{gpu} is designed to do thousands of tasks at the same time at an overall slower rate, where a \gls{cpu} is designed to do tasks one after the other incredibly quickly.

I am testing one of the most computationally expensive \gls{dsp} operations - massive convolution - on a \gls{cpu} and a \gls{gpu}. The goal is to benchmark them and see at what point it's better to use which chip.